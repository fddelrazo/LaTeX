%%%%%%%%%%%%%%%%%%%%%%%%%%%%%%%%%%%%%%%%%%%%%%%%%%%%%%%%%%%%
%% Apéndice B / Código Fuente
%%%%%%%%%%%%%%%%%%%%%%%%%%%%%%%%%%%%%%%%%%%%%%%%%%%%%%%%%%%%

%%%%%%%%%%%%%%%%%%%%%%%%%%%%%%%%%%%%%%%%%%%%%%%%%%%%% Preparación para el paquete listing
\definecolor{gray97}{gray}{.97}
\definecolor{gray75}{gray}{.75}
\definecolor{gray45}{gray}{.45}

\lstset{ frame=Ltb,
  framerule=0pt,
  aboveskip=0.5cm,
  framextopmargin=3pt,
  framexbottommargin=3pt,
  framexleftmargin=0.4cm,
  framesep=0pt,
  rulesep=.4pt,
  backgroundcolor=\color{gray97},
  rulesepcolor=\color{black},
  %
  stringstyle=\ttfamily,
  showstringspaces = false,
  basicstyle=\small\ttfamily,
  commentstyle=\color{gray45},
  keywordstyle=\bfseries,
  %
  numbers=left,
  numbersep=15pt,
  numberstyle=\tiny,
  numberfirstline = false,
  breaklines=true,
}
% Personalizando los ambientes para listing
\lstnewenvironment{listing}[1][]{
  \lstset{#1}\pagebreak[0]
}
{\pagebreak[0]}
\lstdefinestyle{terminal}{			% Terminal
  basicstyle=\scriptsize\bf\ttfamily,
  backgroundcolor=\color{gray75},
}
\lstdefinestyle{C} {				% Código en C
  language=C,
}

%%%%%%%%%%%%%%%%%%%%%%%%%%%%%%%%%%%%%%%%%%%%%%%%%%%%%%%%%%%%
\chapter{Código Fuente}

\section{Explicando un programa}
\lipsum[1-2]

\noindent
Escribe el siguiente programa en un archivo llamado \texttt{hola.c}:

\begin{lstlisting}[style=C]
#include <stdio.h>

int main(int argc, char* argv[]) 
{
    printf("Hola mundo!");
}
\end{lstlisting}

\noindent
Ahora compila usando \texttt{gcc}:

\begin{listing}[style=terminal, numbers=none]
$ gcc  -o hola hola.c
\end{listing}