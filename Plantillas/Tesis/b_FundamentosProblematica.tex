%%%%%%%%%%%%%%%%%%%%%%%%%%%%%%%%%%%%%%%%%%%%%%%%%%%%%%%%%%%%
%% Capítulo 2 / Fundamentos Teóricos y Problemática
%%%%%%%%%%%%%%%%%%%%%%%%%%%%%%%%%%%%%%%%%%%%%%%%%%%%%%%%%%%%
\chapter{Fundamentos Teóricos y Problemática}
\epigrafe{Everyone in this country should learn how to program a computer, because it teaches you how to think.}
              {\textsc{Steve Jobs}}

\section{Ejemplo Ecuación}

\lipsum[1]

\begin{equation}                                                     
\mathrm{e}^{x} = \sum_{k=0}^{\infty} \frac{x^k}{k!} = 
1 + x + \frac{x^2}{2} + \frac{x^3}{6} + \cdots
\label{eq:taylor}
\end{equation}

La ecuación \eqref{eq:taylor} muestra la expansión en serie de Taylor alrededor de $x = 0$ para la función exponencial natural.

\section{Ejemplo Cita Bibliográfica}
\lipsum[3-4].  
Los fundamentos también pueden revisarse en \cite{knuth97}. 

\section{Tercera sección}
\lipsum[5-6]
