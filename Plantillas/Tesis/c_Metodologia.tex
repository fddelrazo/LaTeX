%%%%%%%%%%%%%%%%%%%%%%%%%%%%%%%%%%%%%%%%%%%%%%%%%%%%%%%%%%%%
%% Capítulo 4 / Metodología
%%%%%%%%%%%%%%%%%%%%%%%%%%%%%%%%%%%%%%%%%%%%%%%%%%%%%%%%%%%%
\chapter{Metodología}
\epigrafe{Everyone in this country should learn how to program a computer, because it teaches you how to think.}
              {\textsc{Steve Jobs}}
              
\textcolor{gray}{En este apartado se explican los pasos que se desarrollaron, en cada una de las etapas del trabajo de tesis, se debe tener cuidado de ser congruente. Si es necesario, apoyarse con un esquema a fin de mostrar de manera global el desarrollo del trabajo e ir presentando posteriormente la descripción de cada una de las etapas.}

\section{Primera sección}
\lipsum[1-2]. 
Se deben atender las recomendaciones y buena prácticas indicadas en su seminario de investigación, de tesis, o de
las referencias bibliográficas pertinentes, por ejemplo \cite{Sampieri}. 

\section{Ejemplo Algoritmo}
\lipsum[3]
\lipsum[4].
El análisis de datos se realiza con el Algoritmo \ref{Alg:Algo-01}, descrito en seguida:

\begin{algorithm}[H]
  \SetAlgoLined
  \KwData{this text}
  \KwResult{how to write algorithm with \LaTeX2e }
  initialization\;
  \While{not at end of this document}{
    read current\;
    \eIf{understand}{
      go to next section\;
      current section becomes this one\;
      }{
      go back to the beginning of current section\;
      }
    }
  \caption{How to write algorithms}
  \label{Alg:Algo-01}
\end{algorithm}

\section{Ejemplos Citas Bibliográficas}
\lipsum[1-2].
Actualmente existen las siguiente propuestas de \cite{lopez2013impacto} y \cite{orduna2016revolucion}.


